%%%%%%%%%%%%%%%%%%%%%%%%%%%%%%%%%%%%%%%%%
% Large Colored Title Article
% LaTeX Template
% Version 1.1 (25/11/12)
%
% This template has been downloaded from:
% http://www.LaTeXTemplates.com
%
% Original author:
% Frits Wenneker (http://www.howtotex.com)
%
% License:
% CC BY-NC-SA 3.0 (http://creativecommons.org/licenses/by-nc-sa/3.0/)
%
%%%%%%%%%%%%%%%%%%%%%%%%%%%%%%%%%%%%%%%%%

%----------------------------------------------------------------------------------------
%	PACKAGES AND OTHER DOCUMENT CONFIGURATIONS
%----------------------------------------------------------------------------------------

\documentclass[DIV=calc, paper=a4, fontsize=10pt, twocolumn]{scrartcl}	 % A4 paper and 11pt font size

\usepackage{lipsum} % Used for inserting dummy 'Lorem ipsum' text into the template
\usepackage[english]{babel} % English language/hyphenation
\usepackage[protrusion=true,expansion=true]{microtype} % Better typography
\usepackage{amsmath,amsfonts,amsthm} % Math packages
\usepackage[svgnames]{xcolor} % Enabling colors by their 'svgnames'
\usepackage[hang, small,labelfont=bf,up,textfont=it,up]{caption} % Custom captions under/above floats in tables or figures
\usepackage{booktabs} % Horizontal rules in tables
\usepackage{fix-cm}	 % Custom font sizes - used for the initial letter in the document

\usepackage{sectsty} % Enables custom section titles
\allsectionsfont{\usefont{OT1}{phv}{b}{n}} % Change the font of all section commands

\usepackage{fancyhdr} % Needed to define custom headers/footers
\pagestyle{fancy} % Enables the custom headers/footers
\usepackage{lastpage} % Used to determine the number of pages in the document (for "Page X of Total")

% Headers - all currently empty
\lhead{}
\chead{}
\rhead{}

% Footers
\lfoot{}
\cfoot{}
\rfoot{\footnotesize Page \thepage\ of \pageref{LastPage}} % "Page 1 of 2"

\renewcommand{\headrulewidth}{0.0pt} % No header rule
\renewcommand{\footrulewidth}{0.4pt} % Thin footer rule

\usepackage{lettrine} % Package to accentuate the first letter of the text
\newcommand{\initial}[1]{ % Defines the command and style for the first letter
\lettrine[lines=3,lhang=0.3,nindent=0em]{
\color{DarkGoldenrod}
{\textsf{#1}}}{}}

%----------------------------------------------------------------------------------------
%	TITLE SECTION
%----------------------------------------------------------------------------------------

\usepackage{titling} % Allows custom title configuration

\newcommand{\HorRule}{\color{DarkGoldenrod} \rule{\linewidth}{1pt}} % Defines the gold horizontal rule around the title

\pretitle{\vspace{-30pt} \begin{flushleft} \HorRule \fontsize{50}{50} \usefont{OT1}{phv}{b}{n} \color{DarkRed} \selectfont} % Horizontal rule before the title

\title{Massively Parallel Finite Element Methods} % Your article title

\posttitle{\par\end{flushleft}\vskip 0.5em} % Whitespace under the title

\preauthor{\begin{flushleft}\large \lineskip 0.5em \usefont{OT1}{phv}{b}{sl} \color{DarkRed}} % Author font configuration

\author{Jonathan Hoy } % Your name

\postauthor{\footnotesize \usefont{OT1}{phv}{m}{sl} \color{Black} % Configuration for the institution name
University of Southern California % Your institution

\par\end{flushleft}\HorRule} % Horizontal rule after the title

\date{} % Add a date here if you would like one to appear underneath the title block

%----------------------------------------------------------------------------------------

\begin{document}

\maketitle % Print the title

\thispagestyle{fancy} % Enabling the custom headers/footers for the first page 

%----------------------------------------------------------------------------------------
%	ABSTRACT
%----------------------------------------------------------------------------------------

% The first character should be within \initial{}
\initial{W}\textbf{ith the advent of massively parallel computer architecture, traditional serial algorithms used in the process of formulating, solving, and post-processing Finite Elements need to be updated to reflect current and future changes to computer architecture. From development of sparse parallel linear solvers to domain decomposition techniques to parallel mesh generation, various techniques will be explored and evaluated}

%----------------------------------------------------------------------------------------
%	ARTICLE CONTENTS
%----------------------------------------------------------------------------------------

\section*{Introduction}

Using the finite element method, various physical phenomena in classical physics can be represented within a body/continuum. This continuum can be solid, liquid, gaseous, or anything in between. The only requirement of this method is that the physical phenomena is able to be represented/approximated with a discrete number of points using differential equations and occurs spatially x-y-z space.

One of the biggest problems with trying to solve phenomena such as turbulence is that the theoretical grid sizes and memory requirements needed to get an accurate representation of the domain over time and space $(x,y,z,t)$ can be prohibitively expensive in terms of RAM, Bandwidth, and FLOPS with todays current machines. One possible technique to lessen the impact of this is to use data compression techniques to store vector fields of the temporal derivative in space along with a coarser vector field. Because recent trends in supercomputing suggest that RAM and Bandwidth will become more expensive in relation to FLOPS in the future, algorithms which utilize far less RAM and network bandwidth at the expense of an increase FLOPS with play to the strength of current and future trends in technology.  
As an example, consider a 2-D complex fluid velocity field $\vec{V}$. Pretend that we are using a GPU to process this field. To make the best use of the GPU, first we must consider in what areas it excels at and at which areas it does not.


Strengths of the GPU:

\begin{itemize}
\item Massive Parallel Computation (~1000 cores)
	\subitem High Theoretical FLOPS (Teraflops)
	\subitem Very High Energy Efficiency per Flop
\item Graphics
	\subitem Dedicated Hardware on the GPU
	\subitem Well Established Graphics APIs for compression
\item Bandwidth
	\subitem High Global Memory Bandwidth
	\subitem Programmable Cache 
\end{itemize}

Weaknesses of the GPU:

\begin{itemize}
\item Thread Divergence
\item Low Bandwidth between the GPU Memory and CPU Memory (~5GB/s)
\item Low individual clock speed (~1Ghz)
\end{itemize}

Because the GPU was originally designed for graphics computation, it has many well-established graphics APIS as well as dedicated hardware for graphics purposes. If we were to represent the gradient 
$\nabla \vec{V}$ of a 2d velocity using a compression technique, we could perform integration and decoding on this compressed field to reproduce the velocity field on the GPU reducing CPU and GPU bandwidth at the expense of numerical artifacts being introduced into the velocity field. While this is certainly not desirable, this is partially mitigated by integration which helps to reduce artificial noise introduced by the compression algorithm. Moreover, if we performed additional image filtering on the uncompressed velocity field, we could further reduce artificial noise at the expense of more FLOPS. Moreover, because of the limited global memory on the GPU compared to the CPU, we could store a compressed version of the field that would be a higher in fidelity than an uncompressed version using the same amount of global memory. By uncompressing the data, we would be using FLOPS on the high fidelity decoded field but sending compressed versions to reduce strain on precious system bandwidth.  

In Finite element analysis, one of the most important the differential equations into a linear system of equations given by Equation \ref{eq:Ax_b}
\begin{equation} \label{eq:Ax_b}
\boldmath{A}x = b
\end{equation}

where A is a sparse square matrix of size $n \times n$ and x is a vector of length $n$ For sparse matrices there exists many methods for solving linear systems \cite{bib:Golub}.  

\begin{itemize}
\item Iterative Methods
\item Direct Methods
\end{itemize}  
%------------------------------------------------

\subsection*{Subsection 1}

\lipsum[5] % Dummy text

\begin{itemize}
\item First item in a list 
\item Second item in a list 
\item Third item in a list
\end{itemize}

\lipsum[6] % Dummy text

%------------------------------------------------

\subsection*{Subsection 2}

\lipsum[7] % Dummy text

\begin{table}
\caption{Random table}
\centering
\begin{tabular}{llr}
\toprule
\multicolumn{2}{c}{Name} \\
\cmidrule(r){1-2}
First name & Last Name & Grade \\
\midrule
John & Doe & $7.5$ \\
Richard & Miles & $2$ \\
\bottomrule
\end{tabular}
\end{table}

%------------------------------------------------

\section*{Section 2}

\lipsum[8] % Dummy text

\begin{description}
\item[First] This is the first item
\item[Last] This is the last item
\end{description}

\lipsum[9] % Dummy text

%----------------------------------------------------------------------------------------
%	REFERENCE LIST
%----------------------------------------------------------------------------------------

\begin{thebibliography}{99} % Bibliography - this is intentionally simple in this template

\bibitem[Matrix Computations, 2013]{bib:Golub}
Golub, Gene and Van Loan, Charles (2013). 
\bibitem[Figueredo and Wolf, 2009]{Figueredo:2009dg}
Figueredo, A.~J. and Wolf, P. S.~A. (2009).
\newblock Assortative pairing and life history strategy - a cross-cultural
  study.
\newblock {\em Human Nature}, 20:317--330.
  
 
\end{thebibliography}

%----------------------------------------------------------------------------------------

\end{document}